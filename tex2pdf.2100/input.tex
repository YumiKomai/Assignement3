\documentclass[]{article}
\usepackage{lmodern}
\usepackage{amssymb,amsmath}
\usepackage{ifxetex,ifluatex}
\usepackage{fixltx2e} % provides \textsubscript
\ifnum 0\ifxetex 1\fi\ifluatex 1\fi=0 % if pdftex
  \usepackage[T1]{fontenc}
  \usepackage[utf8]{inputenc}
\else % if luatex or xelatex
  \ifxetex
    \usepackage{mathspec}
  \else
    \usepackage{fontspec}
  \fi
  \defaultfontfeatures{Ligatures=TeX,Scale=MatchLowercase}
  \newcommand{\euro}{€}
\fi
% use upquote if available, for straight quotes in verbatim environments
\IfFileExists{upquote.sty}{\usepackage{upquote}}{}
% use microtype if available
\IfFileExists{microtype.sty}{%
\usepackage{microtype}
\UseMicrotypeSet[protrusion]{basicmath} % disable protrusion for tt fonts
}{}
\usepackage[margin=1in]{geometry}
\usepackage{hyperref}
\PassOptionsToPackage{usenames,dvipsnames}{color} % color is loaded by hyperref
\hypersetup{unicode=true,
            pdftitle={Assignment3},
            pdfborder={0 0 0},
            breaklinks=true}
\urlstyle{same}  % don't use monospace font for urls
\usepackage{color}
\usepackage{fancyvrb}
\newcommand{\VerbBar}{|}
\newcommand{\VERB}{\Verb[commandchars=\\\{\}]}
\DefineVerbatimEnvironment{Highlighting}{Verbatim}{commandchars=\\\{\}}
% Add ',fontsize=\small' for more characters per line
\usepackage{framed}
\definecolor{shadecolor}{RGB}{248,248,248}
\newenvironment{Shaded}{\begin{snugshade}}{\end{snugshade}}
\newcommand{\KeywordTok}[1]{\textcolor[rgb]{0.13,0.29,0.53}{\textbf{{#1}}}}
\newcommand{\DataTypeTok}[1]{\textcolor[rgb]{0.13,0.29,0.53}{{#1}}}
\newcommand{\DecValTok}[1]{\textcolor[rgb]{0.00,0.00,0.81}{{#1}}}
\newcommand{\BaseNTok}[1]{\textcolor[rgb]{0.00,0.00,0.81}{{#1}}}
\newcommand{\FloatTok}[1]{\textcolor[rgb]{0.00,0.00,0.81}{{#1}}}
\newcommand{\ConstantTok}[1]{\textcolor[rgb]{0.00,0.00,0.00}{{#1}}}
\newcommand{\CharTok}[1]{\textcolor[rgb]{0.31,0.60,0.02}{{#1}}}
\newcommand{\SpecialCharTok}[1]{\textcolor[rgb]{0.00,0.00,0.00}{{#1}}}
\newcommand{\StringTok}[1]{\textcolor[rgb]{0.31,0.60,0.02}{{#1}}}
\newcommand{\VerbatimStringTok}[1]{\textcolor[rgb]{0.31,0.60,0.02}{{#1}}}
\newcommand{\SpecialStringTok}[1]{\textcolor[rgb]{0.31,0.60,0.02}{{#1}}}
\newcommand{\ImportTok}[1]{{#1}}
\newcommand{\CommentTok}[1]{\textcolor[rgb]{0.56,0.35,0.01}{\textit{{#1}}}}
\newcommand{\DocumentationTok}[1]{\textcolor[rgb]{0.56,0.35,0.01}{\textbf{\textit{{#1}}}}}
\newcommand{\AnnotationTok}[1]{\textcolor[rgb]{0.56,0.35,0.01}{\textbf{\textit{{#1}}}}}
\newcommand{\CommentVarTok}[1]{\textcolor[rgb]{0.56,0.35,0.01}{\textbf{\textit{{#1}}}}}
\newcommand{\OtherTok}[1]{\textcolor[rgb]{0.56,0.35,0.01}{{#1}}}
\newcommand{\FunctionTok}[1]{\textcolor[rgb]{0.00,0.00,0.00}{{#1}}}
\newcommand{\VariableTok}[1]{\textcolor[rgb]{0.00,0.00,0.00}{{#1}}}
\newcommand{\ControlFlowTok}[1]{\textcolor[rgb]{0.13,0.29,0.53}{\textbf{{#1}}}}
\newcommand{\OperatorTok}[1]{\textcolor[rgb]{0.81,0.36,0.00}{\textbf{{#1}}}}
\newcommand{\BuiltInTok}[1]{{#1}}
\newcommand{\ExtensionTok}[1]{{#1}}
\newcommand{\PreprocessorTok}[1]{\textcolor[rgb]{0.56,0.35,0.01}{\textit{{#1}}}}
\newcommand{\AttributeTok}[1]{\textcolor[rgb]{0.77,0.63,0.00}{{#1}}}
\newcommand{\RegionMarkerTok}[1]{{#1}}
\newcommand{\InformationTok}[1]{\textcolor[rgb]{0.56,0.35,0.01}{\textbf{\textit{{#1}}}}}
\newcommand{\WarningTok}[1]{\textcolor[rgb]{0.56,0.35,0.01}{\textbf{\textit{{#1}}}}}
\newcommand{\AlertTok}[1]{\textcolor[rgb]{0.94,0.16,0.16}{{#1}}}
\newcommand{\ErrorTok}[1]{\textcolor[rgb]{0.64,0.00,0.00}{\textbf{{#1}}}}
\newcommand{\NormalTok}[1]{{#1}}
\usepackage{graphicx,grffile}
\makeatletter
\def\maxwidth{\ifdim\Gin@nat@width>\linewidth\linewidth\else\Gin@nat@width\fi}
\def\maxheight{\ifdim\Gin@nat@height>\textheight\textheight\else\Gin@nat@height\fi}
\makeatother
% Scale images if necessary, so that they will not overflow the page
% margins by default, and it is still possible to overwrite the defaults
% using explicit options in \includegraphics[width, height, ...]{}
\setkeys{Gin}{width=\maxwidth,height=\maxheight,keepaspectratio}
\setlength{\parindent}{0pt}
\setlength{\parskip}{6pt plus 2pt minus 1pt}
\setlength{\emergencystretch}{3em}  % prevent overfull lines
\providecommand{\tightlist}{%
  \setlength{\itemsep}{0pt}\setlength{\parskip}{0pt}}
\setcounter{secnumdepth}{0}

%%% Use protect on footnotes to avoid problems with footnotes in titles
\let\rmarkdownfootnote\footnote%
\def\footnote{\protect\rmarkdownfootnote}

%%% Change title format to be more compact
\usepackage{titling}

% Create subtitle command for use in maketitle
\newcommand{\subtitle}[1]{
  \posttitle{
    \begin{center}\large#1\end{center}
    }
}

\setlength{\droptitle}{-2em}
  \title{Assignment3}
  \pretitle{\vspace{\droptitle}\centering\huge}
  \posttitle{\par}
  \author{}
  \preauthor{}\postauthor{}
  \date{}
  \predate{}\postdate{}



% Redefines (sub)paragraphs to behave more like sections
\ifx\paragraph\undefined\else
\let\oldparagraph\paragraph
\renewcommand{\paragraph}[1]{\oldparagraph{#1}\mbox{}}
\fi
\ifx\subparagraph\undefined\else
\let\oldsubparagraph\subparagraph
\renewcommand{\subparagraph}[1]{\oldsubparagraph{#1}\mbox{}}
\fi

\begin{document}
\maketitle

\subsection{Research Question}\label{research-question}

Twenty years since the end of Apartheid: Did the collapse of Apartheid
play a significant role in reducing racial and social inequality in
South Africa? Is post-apartheid South Africa better off or worse off
than during the apartheid era?

\subsection{Definition of Racial and social
inequality}\label{definition-of-racial-and-social-inequality}

Before conducting the data analysis to find the answer for our research
question, we begin by clarifying the definitions for racial and social
inequality. While social inequality broadly refers to the existence of
unequal opportunities for different social status/positions within a
society, racial inequality can be seen as one of dimensions of social
inequality. It thus indicates the discrimination based on race in access
to socioeconomic opportunities or services. In our research study, we
will specifically look into racial discrimination in terms of
employment, education, and income levels. Because these three indicators
within the capitalism society can be seen as fundamental yet significant
estimators for the quality of human well-being, we decided to include
them. In addition, we will try to identify drivers of unequal income
distribution by controlling possible factors and variables such as
unemployment rate and education level.

\subsection{Literature review}\label{literature-review}

In order for us to bring out more in-depth analysis, we undertook
background researches by examining the past studies written by various
researchers. First of all, according to Leibbrandt,(see Leibbrandt
(n.d.)), Since the fall of Apartheid(1993\textasciitilde{}2008), overall
(include all races) income inequality has increased and it was mainly
caused by huge inequality within black African community in
South-Africa. We chose this article as the first reference since it has
been cited the most for the South-African Inequality Study. Second
research literature is ``One Kind of Freedom: Poverty Dynamics in
Post-Apartheid South Africa,'' which explores whether the legacy of
apartheid in terms of inequality and human insecurity has been
superseded by looking at the dynamics of post-apartheid income
distribution based on the data from national household surveys. ``Income
and Non-income Inequality in Post-Apartheid South Africa: What are the
Drivers and Possible Policy Interventions?'' identifies the drivers of
the reproduction of inequality in post-apartheid South Africa and argues
that there had a continuous increase in inequality, strongly indicating
that South African is now the one of the most consistently unequal
economy in the world. Fourth background research literature is ``Poverty
and Well-being in Post-Apartheid South Africa: An Overview of Data,
Outcomes and Policy.'' While this study provides an overview of poverty
and well-being of South African during the first decade of
post-apartheid, it argues that the first ten years after the end of
Apartheid has rather displayed increase in income inequality and
unemployment rates. ``Crime and local inequality in South Africa''
examines the effects of local inequality and violent crime in South
Africa in the post-apartheid era and claims that racial heterogeneity is
highly correlated with all types of crime. Lastly, ``Poverty and
Inequality Dynamics in South Africa: Post-apartheid Developments in the
Light of the Long-Run Legacy'' makes a claim that the bottom half of the
income distribution and poverty has been dominated by these black South
Africans.

\subsection{Data Gathering}\label{data-gathering}

Closely having studied the past researches, we found that most of
researchers made opposite conclusions to ours in regard to the effects
of post-apartheid on the qualiaty of life in South Africa. We therefore
want to test our hypothesis based on the following data analysis and
compare with the past studies.

We found the data of monthly earnings among races and gender. We tried
to scraping the data from the website.

\begin{Shaded}
\begin{Highlighting}[]
\NormalTok{URL <-}\StringTok{ 'http://businesstech.co.za/news/wealth/131524/this-is-the-average-salary-in-south-africa-by-race-and-industry/'}

\NormalTok{RaceEarningsTable <-}\StringTok{ }\NormalTok{URL %>%}\StringTok{ }\KeywordTok{read_html}\NormalTok{() %>%}
\StringTok{                    }\KeywordTok{html_nodes}\NormalTok{(}\StringTok{'#container > div.content_holder > div.content > div.post_single > div.post_content > div:nth-child(11) > table'}\NormalTok{) %>%}
\StringTok{                    }\KeywordTok{html_table}\NormalTok{() %>%}\StringTok{ }
\StringTok{                    }\NormalTok{as.data.frame}
\NormalTok{RaceEarningsTable}
\end{Highlighting}
\end{Shaded}

\begin{verbatim}
##              X1     X2     X3       X4     X5     X6       X7
## 1               Median Median   Median   Mean   Mean     Mean
## 2          Race   2003   2012 Increase   2003   2012 Increase
## 3         White 14 468 16 581      15% 11 249 11 991       7%
## 4  Asian/Indian  7 825 11 701      50%  5 264  8 993      60%
## 5      Coloured  4 241  7 058      66%  2 437  3 897      60%
## 6 Black African  4 059  5 445      34%  2 437  2 998      23%
\end{verbatim}

\begin{Shaded}
\begin{Highlighting}[]
\NormalTok{URL <-}\StringTok{ 'http://businesstech.co.za/news/wealth/131524/this-is-the-average-salary-in-south-africa-by-race-and-industry/'}

\NormalTok{GenderEarningsTable <-}\StringTok{ }\NormalTok{URL %>%}\StringTok{ }\KeywordTok{read_html}\NormalTok{() %>%}
\StringTok{                    }\KeywordTok{html_nodes}\NormalTok{(}\StringTok{'#container > div.content_holder > div.content > div.post_single > div.post_content > div:nth-child(13) > table'}\NormalTok{) %>%}
\StringTok{                    }\KeywordTok{html_table}\NormalTok{() %>%}\StringTok{ }
\StringTok{                    }\NormalTok{as.data.frame}
\NormalTok{GenderEarningsTable}
\end{Highlighting}
\end{Shaded}

\begin{verbatim}
##       X1     X2     X3       X4    X5    X6       X7
## 1        Median Median   Median  Mean  Mean     Mean
## 2   Race   2003   2012 Increase  2003  2012 Increase
## 3   Male  5 963  8 299      39% 3 375 4 317      28%
## 4 Female  4 849  6 399      32% 2 435 3 118      28%
\end{verbatim}

\subsection{Data Cleaning and Merging}\label{data-cleaning-and-merging}

In this section, we will try to clean the data so that they can be
statistical analysed.

Firstly, we use command ``summary'' to investigate the structure (class
of variables, number of vectors) of data frames we got in the previous
section.

\begin{Shaded}
\begin{Highlighting}[]
\KeywordTok{summary}\NormalTok{(RaceEarningsTable)}
\end{Highlighting}
\end{Shaded}

\begin{verbatim}
##       X1                 X2                 X3           
##  Length:6           Length:6           Length:6          
##  Class :character   Class :character   Class :character  
##  Mode  :character   Mode  :character   Mode  :character  
##       X4                 X5                 X6           
##  Length:6           Length:6           Length:6          
##  Class :character   Class :character   Class :character  
##  Mode  :character   Mode  :character   Mode  :character  
##       X7           
##  Length:6          
##  Class :character  
##  Mode  :character
\end{verbatim}

\begin{Shaded}
\begin{Highlighting}[]
\KeywordTok{summary}\NormalTok{(GenderEarningsTable)}
\end{Highlighting}
\end{Shaded}

\begin{verbatim}
##       X1                 X2                 X3           
##  Length:4           Length:4           Length:4          
##  Class :character   Class :character   Class :character  
##  Mode  :character   Mode  :character   Mode  :character  
##       X4                 X5                 X6           
##  Length:4           Length:4           Length:4          
##  Class :character   Class :character   Class :character  
##  Mode  :character   Mode  :character   Mode  :character  
##       X7           
##  Length:4          
##  Class :character  
##  Mode  :character
\end{verbatim}

As shown, every variables has a class of ``characters'' even though it
represents numerical data.

The data we want to have is the mean of earnings among races and gender
in 2003, 2012.

Firstly, we make TimeVector and IndivisualVector to labeling the data.

\begin{Shaded}
\begin{Highlighting}[]
\NormalTok{TimeVector <-}\StringTok{ }\KeywordTok{c}\NormalTok{(}\DecValTok{2003}\NormalTok{,}\DecValTok{2012}\NormalTok{) }\CommentTok{#numerical vector}
\NormalTok{IndivisualVector <-}\StringTok{ }\KeywordTok{c}\NormalTok{(}\StringTok{"Male"}\NormalTok{,}\StringTok{"Female"}\NormalTok{,}\StringTok{"White"}\NormalTok{,}\StringTok{"Asian/Indian"}\NormalTok{,}\StringTok{"Coloured"}\NormalTok{,}\StringTok{"BlackAfrican"}\NormalTok{) }\CommentTok{#character vector}
\end{Highlighting}
\end{Shaded}

Then, we try to convert character vector to numerical vector.

\begin{Shaded}
\begin{Highlighting}[]
\NormalTok{male2003 <-}\StringTok{ }\KeywordTok{as.numeric}\NormalTok{(}\KeywordTok{gsub}\NormalTok{(}\StringTok{"([0-9]+).*$"}\NormalTok{, }\StringTok{"}\CharTok{\textbackslash{}\textbackslash{}}\StringTok{1"}\NormalTok{, }\KeywordTok{str_replace_all}\NormalTok{(GenderEarningsTable$X5[}\DecValTok{3}\NormalTok{], }\KeywordTok{fixed}\NormalTok{(}\StringTok{" "}\NormalTok{), }\StringTok{""}\NormalTok{)))}
\KeywordTok{is.numeric}\NormalTok{(male2003)}
\end{Highlighting}
\end{Shaded}

\begin{verbatim}
## [1] TRUE
\end{verbatim}

\begin{Shaded}
\begin{Highlighting}[]
\NormalTok{male2003}
\end{Highlighting}
\end{Shaded}

\begin{verbatim}
## [1] 3375
\end{verbatim}

As I shown above, the character variable successfully converted to
numerical variable. Then, we make function which conduct this sequence.

\begin{Shaded}
\begin{Highlighting}[]
\NormalTok{Converter <-}\StringTok{ }\NormalTok{function(x)\{}
\NormalTok{y <-}\StringTok{ }\KeywordTok{as.numeric}\NormalTok{(}\KeywordTok{gsub}\NormalTok{(}\StringTok{"([0-9]+).*$"}\NormalTok{, }\StringTok{"}\CharTok{\textbackslash{}\textbackslash{}}\StringTok{1"}\NormalTok{, }\KeywordTok{str_replace_all}\NormalTok{(x, }\KeywordTok{fixed}\NormalTok{(}\StringTok{" "}\NormalTok{), }\StringTok{""}\NormalTok{)))}
\KeywordTok{return}\NormalTok{(y)}
\NormalTok{\}}
\NormalTok{test <-}\StringTok{ }\KeywordTok{Converter}\NormalTok{(}\DataTypeTok{x =} \NormalTok{GenderEarningsTable$X5[}\DecValTok{3}\NormalTok{])}
\KeywordTok{is.numeric}\NormalTok{(test)}
\end{Highlighting}
\end{Shaded}

\begin{verbatim}
## [1] TRUE
\end{verbatim}

\begin{Shaded}
\begin{Highlighting}[]
\NormalTok{test}
\end{Highlighting}
\end{Shaded}

\begin{verbatim}
## [1] 3375
\end{verbatim}

Then, we can apply this function to all data.

\begin{Shaded}
\begin{Highlighting}[]
\CommentTok{#definition of vector}
\NormalTok{Earnings2003 <-}\StringTok{ }\KeywordTok{c}\NormalTok{(}\DecValTok{0}\NormalTok{,}\DecValTok{0}\NormalTok{,}\DecValTok{0}\NormalTok{,}\DecValTok{0}\NormalTok{,}\DecValTok{0}\NormalTok{,}\DecValTok{0}\NormalTok{)}
\NormalTok{Earnings2012 <-}\StringTok{ }\KeywordTok{c}\NormalTok{(}\DecValTok{0}\NormalTok{,}\DecValTok{0}\NormalTok{,}\DecValTok{0}\NormalTok{,}\DecValTok{0}\NormalTok{,}\DecValTok{0}\NormalTok{,}\DecValTok{0}\NormalTok{)}

\CommentTok{#GenderEarnings}
\NormalTok{for(i in }\DecValTok{3}\NormalTok{:}\DecValTok{4}\NormalTok{)\{}
  \NormalTok{Earnings2003[i}\DecValTok{-2}\NormalTok{] =}\StringTok{ }\KeywordTok{Converter}\NormalTok{(}\DataTypeTok{x =} \NormalTok{GenderEarningsTable$X5[i])}
  \NormalTok{Earnings2012[i}\DecValTok{-2}\NormalTok{] =}\StringTok{ }\KeywordTok{Converter}\NormalTok{(}\DataTypeTok{x =} \NormalTok{GenderEarningsTable$X6[i])}
\NormalTok{\}}
\CommentTok{#RaceEarnings}
\NormalTok{for(i in }\DecValTok{3}\NormalTok{:}\DecValTok{6}\NormalTok{)\{}
  \NormalTok{Earnings2003[i] =}\StringTok{ }\KeywordTok{Converter}\NormalTok{(}\DataTypeTok{x =} \NormalTok{RaceEarningsTable$X5[i])}
  \NormalTok{Earnings2012[i] =}\StringTok{ }\KeywordTok{Converter}\NormalTok{(}\DataTypeTok{x =} \NormalTok{RaceEarningsTable$X6[i])}
\NormalTok{\}}
\NormalTok{Earnings2003}
\end{Highlighting}
\end{Shaded}

\begin{verbatim}
## [1]  3375  2435 11249  5264  2437  2437
\end{verbatim}

\begin{Shaded}
\begin{Highlighting}[]
\NormalTok{Earnings2012}
\end{Highlighting}
\end{Shaded}

\begin{verbatim}
## [1]  4317  3118 11991  8993  3897  2998
\end{verbatim}

\begin{Shaded}
\begin{Highlighting}[]
\NormalTok{preEarnings <-}\StringTok{ }\KeywordTok{data.frame}\NormalTok{(IndivisualVector,Earnings2003, Earnings2012)}
\NormalTok{preEarnings}
\end{Highlighting}
\end{Shaded}

\begin{verbatim}
##   IndivisualVector Earnings2003 Earnings2012
## 1             Male         3375         4317
## 2           Female         2435         3118
## 3            White        11249        11991
## 4     Asian/Indian         5264         8993
## 5         Coloured         2437         3897
## 6     BlackAfrican         2437         2998
\end{verbatim}

The preEarnings is messy data.

So we are going to transform it into tidy data.

\begin{Shaded}
\begin{Highlighting}[]
\KeywordTok{library}\NormalTok{(tidyr)}
\end{Highlighting}
\end{Shaded}

\begin{verbatim}
## Warning: package 'tidyr' was built under R version 3.3.2
\end{verbatim}

\begin{Shaded}
\begin{Highlighting}[]
\NormalTok{Earnings <-}\StringTok{ }\KeywordTok{gather}\NormalTok{(preEarnings, time, mean, Earnings2003:Earnings2012)}
\NormalTok{Earnings}
\end{Highlighting}
\end{Shaded}

\begin{verbatim}
##    IndivisualVector         time  mean
## 1              Male Earnings2003  3375
## 2            Female Earnings2003  2435
## 3             White Earnings2003 11249
## 4      Asian/Indian Earnings2003  5264
## 5          Coloured Earnings2003  2437
## 6      BlackAfrican Earnings2003  2437
## 7              Male Earnings2012  4317
## 8            Female Earnings2012  3118
## 9             White Earnings2012 11991
## 10     Asian/Indian Earnings2012  8993
## 11         Coloured Earnings2012  3897
## 12     BlackAfrican Earnings2012  2998
\end{verbatim}

We suceeded to make the numerical vector showing the earnings among
races and genders.

\subsection{Conduct basic descriptive
statistics}\label{conduct-basic-descriptive-statistics}

The data we gathered in previous section partialy statisticaly analysed
(mean and median are already calcurated). In this section, we try to
figure out the trend of inequality graphycally by using descriptive
statistics.

\subsection{Briefly discribing}\label{briefly-discribing}

\section*{References}\label{references}
\addcontentsline{toc}{section}{References}

\hypertarget{refs}{}
\hypertarget{ref-Leibbrandt}{}
Leibbrandt, et al., \textless{}!--// //--\textgreater{} M. n.d. ``Trends
in South African Income Distribution and Poverty Since the Fall of
Apartheid.'' OECD Publishing.
doi:\href{https://doi.org/http://dx.doi.org/10.1787/5kmms0t7p1ms-en}{http://dx.doi.org/10.1787/5kmms0t7p1ms-en}.

\end{document}
